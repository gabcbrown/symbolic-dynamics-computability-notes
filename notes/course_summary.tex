% imports
\documentclass[11pt, reqno]{amsart}
\usepackage[margin=1in]{geometry} 

% \geometry{letterpaper}       
%\geometry{landscape}                % Activate for for rotated page geometry
\usepackage[parfill]{parskip}    % Deactivate to begin paragraphs with an indent rather than an empty line
\setlength{\parindent}{20pt}   
\usepackage{amsfonts, amscd, amssymb, amsthm, amsmath}
\usepackage{pdfsync}  %leaves makers for tex searching
\usepackage{enumerate}
\usepackage{multicol}
\usepackage[pdftex,bookmarks]{hyperref}
\usepackage{biblatex}
\usepackage{graphicx}				   % For importing photos. 
\usepackage{caption}           % centering captions
\usepackage{float}
\setcounter{tocdepth}{3}
\setcounter{secnumdepth}{3}
\let\oldtocsection=\tocsection

\let\oldtocsubsection=\tocsubsection

\let\oldtocsubsubsection=\tocsubsubsection

\DeclareFieldFormat{postnote}{#1}

\renewcommand{\tocsection}[2]{\hspace{0em}\oldtocsection{#1}{#2}}
\renewcommand{\tocsubsection}[2]{\hspace{1em}\oldtocsubsection{#1}{#2}}
\renewcommand{\tocsubsubsection}[2]{\hspace{2em}\oldtocsubsubsection{#1}{#2}}

\addbibresource{bib.bib} % Import the bibliography
\graphicspath{ {./images/} }


\hypersetup{
  colorlinks=true,
  allcolors=blue
}

%%% Theorems %%%--------------------------------------------------------- 
\theoremstyle{plain}
    \newtheorem{thm}{Theorem}[section]
    % \numberwithin{thm}{subsection}
    \newtheorem{lemma}[thm]{Lemma}
    \newtheorem{prop}[thm]{Proposition}
    \newtheorem{cor}[thm]{Corollary}
\theoremstyle{definition}
    \newtheorem{defn}[thm]{Definition}
    \newtheorem{remark}{Remark}


%%% Environments %%%--------------------------------------------------------- 
\newenvironment{ans}{\color{black}\medskip \paragraph*{\emph{Answer}.}}{\hfill \break  $~\!\!$ \dotfill \medskip }
\newenvironment{sketch}{\medskip \paragraph*{\emph{Proof sketch}.}}{ \medskip }
\newenvironment{summary}{\medskip \paragraph*{\emph{Summary}.}}{  \hfill \break  \rule{1.5cm}{0.4pt} \medskip }
\newcommand\Ans[1]{\color{black}\hfill \emph{Answer:} {#1}}


%%% Pictures %%%--------------------------------------------------------- 
%%% If you need to draw pictures, tikzpicture is one good option. Here are some basic things I always use:
\usepackage{tikz}
\usetikzlibrary{arrows, decorations.text,calc,arrows.meta}
\tikzstyle{V}=[draw, fill =black, circle, inner sep=0pt, minimum size=2pt]
\tikzstyle{bV}=[draw, fill =black, circle, inner sep=0pt, minimum size=3.5pt]
\newcommand\TikZ[1]{\begin{matrix}\begin{tikzpicture}#1\end{tikzpicture}\end{matrix}}
\definecolor{r}{RGB}{227, 242, 253}
\definecolor{o}{RGB}{187, 222, 251}
\definecolor{y}{RGB}{144, 202, 249}
\definecolor{g}{RGB}{66, 165, 245}
\definecolor{b}{RGB}{30, 136, 229}

%%% Color  %%%---------------------------------------------------------
\usepackage{color}
\newcommand{\blue}[1]{{\color{blue}#1}}
\newcommand{\NOTE}[1]{{\color{blue}#1}}
\newcommand{\MOVED}[1]{{\color{gray}#1}}


%%% Alphabets %%%---------------------------------------------------------
%%% Some shortcuts for my commonly used special alphabets and characters.
\def\cA{\mathcal{A}}\def\cB{\mathcal{B}}\def\cC{\mathcal{C}}\def\cD{\mathcal{D}}\def\cE{\mathcal{E}}\def\cF{\mathcal{F}}\def\cG{\mathcal{G}}\def\cH{\mathcal{H}}\def\cI{\mathcal{I}}\def\cJ{\mathcal{J}}\def\cK{\mathcal{K}}\def\cL{\mathcal{L}}\def\cM{\mathcal{M}}\def\cN{\mathcal{N}}\def\cO{\mathcal{O}}\def\cP{\mathcal{P}}\def\cQ{\mathcal{Q}}\def\cR{\mathcal{R}}\def\cS{\mathcal{S}}\def\cT{\mathcal{T}}\def\cU{\mathcal{U}}\def\cV{\mathcal{V}}\def\cW{\mathcal{W}}\def\cX{\mathcal{X}}\def\cY{\mathcal{Y}}\def\cZ{\mathcal{Z}}

\def\AA{\mathbb{A}} \def\BB{\mathbb{B}} \def\CC{\mathbb{C}} \def\DD{\mathbb{D}} \def\EE{\mathbb{E}} \def\FF{\mathbb{F}} \def\GG{\mathbb{G}} \def\HH{\mathbb{H}} \def\II{\mathbb{I}} \def\JJ{\mathbb{J}} \def\KK{\mathbb{K}} \def\LL{\mathbb{L}} \def\MM{\mathbb{M}} \def\NN{\mathbb{N}} \def\OO{\mathbb{O}} \def\PP{\mathbb{P}} \def\QQ{\mathbb{Q}} \def\RR{\mathbb{R}} \def\SS{\mathbb{S}} \def\TT{\mathbb{T}} \def\UU{\mathbb{U}} \def\VV{\mathbb{V}} \def\WW{\mathbb{W}} \def\XX{\mathbb{X}} \def\YY{\mathbb{Y}} \def\ZZ{\mathbb{Z}}  

\def\fa{\mathfrak{a}} \def\fb{\mathfrak{b}} \def\fc{\mathfrak{c}} \def\fd{\mathfrak{d}} \def\fe{\mathfrak{e}} \def\ff{\mathfrak{f}} \def\fg{\mathfrak{g}} \def\fh{\mathfrak{h}} \def\fj{\mathfrak{j}} \def\fk{\mathfrak{k}} \def\fl{\mathfrak{l}} \def\fm{\mathfrak{m}} \def\fn{\mathfrak{n}} \def\fo{\mathfrak{o}} \def\fp{\mathfrak{p}} \def\fq{\mathfrak{q}} \def\fr{\mathfrak{r}} \def\fs{\mathfrak{s}} \def\ft{\mathfrak{t}} \def\fu{\mathfrak{u}} \def\fv{\mathfrak{v}} \def\fw{\mathfrak{w}} \def\fx{\mathfrak{x}} \def\fy{\mathfrak{y}} \def\fz{\mathfrak{z}}
\def\fgl{\mathfrak{gl}}  \def\fsl{\mathfrak{sl}}  \def\fso{\mathfrak{so}}  \def\fsp{\mathfrak{sp}}  
\def\GL{\mathrm{GL}} \def\SL{\mathrm{SL}}  \def\SP{\mathrm{SP}} \def\O{\mathrm{O}}

\def\<{\langle} \def\>{\rangle}
\usepackage{mathabx}
\def\acts{\lefttorightarrow}
\def\ad{\mathrm{ad}} 
\def\Aut{\mathrm{Aut}}
\def\Ann{\mathrm{Ann}}
\def\dim{\mathrm{dim}} 
\def\End{\mathrm{End}} 
\def\ev{\mathrm{ev}} 
\def\Fr{\mathcal{F}\mathrm{r}}
\def\half{\hbox{$\frac12$}}
\def\Hom{\mathrm{Hom}} 
\def\id{\mathrm{id}} 
\def\sgn{\mathrm{sgn}}  
\def\supp{\mathrm{supp}}  
\def\Tor{\mathrm{Tor}}
\def\tr{\mathrm{tr}} 
\def\vep{\varepsilon}
\def\f{\varphi}
\def\im{\mathrm{im}}
\def\st{\;|\;}
\def\ses{0\rightarrow X \xrightarrow{f} Y \xrightarrow{g} Z \rightarrow 0}


\def\Obj{\mathrm{Obj}}
\def\normeq{\unlhd}
\def\Set{{\cS\mathrm{et}}}
\def\Fin{{\cF\mathrm{inSet}}}
\def\Set{{\cS\mathrm{et}}}
\def\Grp{{\cG\mathrm{rp}}}
\def\Ab{{\cA\mathrm{b}}}
\def\Mod{{\cM\mathrm{od}}}
\def\ab{\mathrm{ab}}
\def\lcm{\mathrm{lcm}}
\def\ZZn{\ZZ/n\ZZ}
\def\CS3{\CC S_3}



\newcommand{\Hint}[1]{\hfill{\small [\emph{Hint:} {#1}]}}
\newcommand\Cmd[1]{$\backslash$\texttt{#1}}
\newcommand{\SES}[5]{0 \hookrightarrow {#1} \xrightarrow{{#4}} {#2} \xrightarrow{{#5}} {#3} \rightarrow 0}

%%%%%%%%%%%%%%%%%%%%%%%%%%%%%% 
%%%%%%%%%%%%%%%%%%%%%%%%%%%%%%

\title{Course Summary}
\author{Gabriela Brown\\ Math B49 \\ 05/23/2022}

\begin{document}
\maketitle

 {
  \setlength{\parskip}{4.4pt}   
  \tableofcontents
 }


\section{Introduction}
TODO

\section{Fundamentals of Shift Spaces}
% \cite{lind-marc}[Ch. 1, Ch. 2]

% Key realizations: 
% \begin{itemize}
% \item The topology (open sets) of a shift space has a basis in cylinder sets. It is thus natural to define a shift space by forbidden words, because we need the shift space to be closed, and the forbidden words give us a countable number open sets that are excluded, the union of which is open, and the complement of which is closed.

% \item Why do we bother with progressive overlap? (i.e. using Nth higher block shift vs Nth higher power shift) Both are shift spaces, but the former are what give us ``sliding block codes" (through factoring?) which when they are invertible are what give us topological conjugacies between shift spaces. Higher block shifts come up later in the Finite Coding Theorem to get around a particular entropy requirement. 

% \item Is 0 needed for the closure?
% \end{itemize}

% \subsection*{1.1}

% \subsection*{1.2}

% defs: 
% shift space 
% forbidden blocks
% subshift
% shift invariance 
% shift map

% give "word definition" and forbidden blocks definition
% e.g. golden mean shift
% e.g. even shift
% e.g. run length limited shift
% e.g. S-gap shift
% e.g. a shift of finite type
% e.g. context free shift
% e.g. failure of closure 

% \subsection*{1.3}
% defs: 
% language
% irreducible 

% prop: correspondence between language and shift space
% // TODO: this can be intuitive, not formal

% \subsection*{1.4}
% defs:
% progressive overlap 
% Nth higher block shift (and notation)
% Nth higher power shift (and notation )

% prop: higher block shift also shift space 
% prop: Nth higher power shift also a shift

% e.g. golden mean shift 

% \subsection*{1.5}
% defs: 
% block map 
% sliding block code with memory m and anticipation n
% topologically conjugate 

% some e.gs. 
% also golden mean 

% prop: sliding block code diagram commutes 
% prop: dynamical properties preserved by sliding block code 
% prop: existence of a factoring of a sliding block code through a higher block shift 
% thm: image under a sliding block code is a shift space 
% thm: invertible sliding block codes are conjugacies 

% \subsection*{2.1}
% defs: 
% shift of finite type 
% M-step (memory M)

% e.g. full shift, golden mean, 3 letter alphabet example from before
% n.e.g. even shift

% thm: conjugate to a shift of finite type => shift of finite type

% ... rest of chapter
% graphs of shifts 
% state splitting 
% data storage 

% ch 3 - sofic shifts 


% questions
% - is the centering condition on a sliding block code equivalent to it being continuous 
%   coding is always continuous based on the metric on the shift space 
%   it's expanding, points are pushed away every iteration 
%   don't worry too much about sliding block codes 
% - example of a full shift code that is not a sliding block code 

% - e.g. 1.2.6, do you need 10^k1 or just 0^k, do you need 0 in the set to get closure?
% closed - we want to do dynamics on a compact phase space
% it's harder to define entropy on non-compact subsets 
% ergodic theory works poorly if you don't have a compact phase space 
% if you don't have compactness, you might not have an invariant measure 

% coded maps and sfts from dynamics are basically the same, because even if you don't have the conjugacy you still have the factor (finite to 1)

% every sofic shift is a factor of an sft 
% every sft is a sofic shift 
% not every sofic shift is an sft 
% finite of 1.2.6 is a sofic shift 
% co-finite case of 1.2.6 is an sft 

% - if you have a lift of a blockmap through a higher alphabet that gives you a 1-block code, why doesn't this recoding not in general give you a 1-block coding for the inverse? Also, what does it mean for the inverse?

\subsection{Coded Shifts}\cite[03/22]{wolf}

\subsection{Topological Entropy in Shift Spaces}\cite[03/02]{wolf}

\section{Computability of shift spaces}

\subsection{Introduction to Computability}\cite[03/08, 03/15]{wolf}   

check notes on computability

\subsection{Computability of Sofic Shifts and SFTs}\cite[03/15]{wolf}  

check notes on computability

\subsection{Crash course in Measure Theory}\cite[03/08]{wolf}  

check notes on computability

\subsection{Crash Course in Ergodic Theory}\cite[04/19]{wolf}

\subsection{Connecting Topological and Measure Theoretic Entropy}\cite[04/26, 04/19]{wolf}

\subsection{Entropy of Coded Shifts}\cite[05/01]{wolf}

\printbibliography




\end{document}
